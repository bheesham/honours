\documentclass[12pt,oneside]{report}
\usepackage{hyperref}
\usepackage[parfill]{parskip}
\usepackage[table]{xcolor}

\usepackage{carleton/titlepage}
\usepackage{carleton/abstract}
\usepackage{carleton/acks}
\usepackage{carleton/tocandlists}

\providecommand\phantomsection{}

%%%%%%%%%%%%%%%%%%%%%%%%%%%%%%%%%%%%%%%%%%%%%%%%%%%%%%%%%%%%%%%%%%%%%%%%%%%%%%%
%%    Thesis information
%%%%%%%%%%%%%%%%%%%%%%%%%%%%%%%%%%%%%%%%%%%%%%%%%%%%%%%%%%%%%%%%%%%%%%%%%%%%%%%

\title{HotSwapTLS}
\author{Bheesham Persaud}
\supervisor{Dr. Anil Somayaji, School of Computer Science}
\date{13 April 2016}

%%%%%%%%%%%%%%%%%%%%%%%%%%%%%%%%%%%%%%%%%%%%%%%%%%%%%%%%%%%%%%%%%%%%%%%%%%%%%%%
%%    Custom settings
%%%%%%%%%%%%%%%%%%%%%%%%%%%%%%%%%%%%%%%%%%%%%%%%%%%%%%%%%%%%%%%%%%%%%%%%%%%%%%%

\linespread{1.25}

\begin{document}
%%%%%%%%%%%%%%%%%%%%%%%%%%%%%%%%%%%%%%%%%%%%%%%%%%%%%%%%%%%%%%%%%%%%%%%%%%%%%%%
%%    Title page
%%%%%%%%%%%%%%%%%%%%%%%%%%%%%%%%%%%%%%%%%%%%%%%%%%%%%%%%%%%%%%%%%%%%%%%%%%%%%%%

\cartitle

%%%%%%%%%%%%%%%%%%%%%%%%%%%%%%%%%%%%%%%%%%%%%%%%%%%%%%%%%%%%%%%%%%%%%%%%%%%%%%%
%%    Abstract
%%%%%%%%%%%%%%%%%%%%%%%%%%%%%%%%%%%%%%%%%%%%%%%%%%%%%%%%%%%%%%%%%%%%%%%%%%%%%%%

\begin{abstract}
	% @TODO
	The purpose of the HotSwapTLS project is to create a method in which
	applications can use different implementation of Transport Layer
	Security (TLS) without requiring recompilation. The TLS implementations
	that we will support include: OpenSSL~\cite{website:openssl},
	LibreSSL~\cite{website:libressl}, and
	BoringSSL~\cite{website:boringssl}.

	This will be achieved by building applications which link against a stub
	library, $libhst$. A minimal TLS API will be exposed by different
	libraries which can be loaded at runtime. Each library will will use a
	different TLS implementation as it's backend.

	Additionally, $hst$, a tool to easily manage different TLS
	implementations and their versions will be created. This is required as
	the TLS implementations are often not compatible with each other --
	installing one overwrites the other, as is the case with LibreSSL and
	OpenSSL.
\end{abstract}

%%%%%%%%%%%%%%%%%%%%%%%%%%%%%%%%%%%%%%%%%%%%%%%%%%%%%%%%%%%%%%%%%%%%%%%%%%%%%%%
%%    Acknowledgements
%%%%%%%%%%%%%%%%%%%%%%%%%%%%%%%%%%%%%%%%%%%%%%%%%%%%%%%%%%%%%%%%%%%%%%%%%%%%%%%

\begin{caracks}
	% @TODO
	I would like to thank my supervisor, Anil Somayaji, for all of his
	guidance with this project.

	I grateful to my parents, Lelawattie and Shradanand Persaud lending
	their advice and their support while pursuing my degree.
\end{caracks}

%%%%%%%%%%%%%%%%%%%%%%%%%%%%%%%%%%%%%%%%%%%%%%%%%%%%%%%%%%%%%%%%%%%%%%%%%%%%%%%
%%    Table of Contents, List of Figures, List of Tables
%%%%%%%%%%%%%%%%%%%%%%%%%%%%%%%%%%%%%%%%%%%%%%%%%%%%%%%%%%%%%%%%%%%%%%%%%%%%%%%

\cartocandlists

%%%%%%%%%%%%%%%%%%%%%%%%%%%%%%%%%%%%%%%%%%%%%%%%%%%%%%%%%%%%%%%%%%%%%%%%%%%%%%%
%%    Paper substance
%%%%%%%%%%%%%%%%%%%%%%%%%%%%%%%%%%%%%%%%%%%%%%%%%%%%%%%%%%%%%%%%%%%%%%%%%%%%%%%

\chapter{Introduction}

Modern computers are often connected to large networks to exchange information.
The majority of the software required to make the exchange of information run
smoothly require all participants to, more or less, implement the same
networking protocols. However, clients must be wary of potential malicious
computers on their network. To mitigate the dangers of malicious computers,
cryptographic protocols are used to protect clients' information while in
transit over a network.

We will be analyzing implementations of one such cryptographic protocol, namely
Transport Security Layer (TLS) -- previously known as Secure Socket Layer (SSL).
The specification for the TLS protocol is long and complex, making it
notoriously hard to correctly implement.

There are multiple security vulnerabilities reported in implementations of the
TLS protocol each year. In 2014 and 2015 the OpenSSL project reported a total of
54 Common Vulnerabilities and Exposures (CVE)~\cite{website:opensslsec}, of
which the infamous Heartbleed bug was included. The announcement of the
Heartbleed bug in OpenSSL caused many community driven forks to be created,
including LibreSSL and BoringSSL. LibreSSL and BoringSSL are forks,
respectively, managed by the OpenBSD developers and by Google.

While many changes and improvements from both BoringSSL and LibreSSL are
incorporated back into OpenSSL, there are significant differences between the
three projects -- this can be attributed to each of their respective goals. 

LibreSSL is developed with the goal of ``... modernizing the codebase, improving
security, and applying best practice development
processes''~\cite{website:libressl}. In the case of BoringSSL, although it ``...
is an open source project, it is not intended for general use, as OpenSSL is. We
don\'t recommend that third parties depend upon it. Doing so is likely to be
frustrating because there are no guarantees of API or ABI
stability.''~\cite{website:boringssl}

Our objective is to create an application which uses best parts of each library.
This paper documents our progress towards consolidating the different
implementations of TLS that OpenSSL, BoringSSL, and LibreSSL offer into one
application. Our design goals and methodology are briefly discussed in
section~\ref{sec:design}; our implementation details and issues we encountered
are outlined in section~\ref{sec:impl}; our evaluation of the implementation of
our aforementioned design goals is presented in section~\ref{sec:eval}; and our
conclusion in section~\ref{sec:con}.

\cleardoublepage

\chapter{Design}
\label{sec:design}

The goal of our project is to take the best parts of each library -- OpenSSL,
BoringSSL, and LibreSSL -- and combine them such a client, at run time, would be
able to mix and match features from any library. The ability to mix and match
features gains us an interesting property: resistance to attacks on any one
specific library. In other words, we could create a program which leverages
OpenSSL, BoringSSL, and LibreSSL without being completely vulnerable to an
exploit found in only one of them.

We believe our goal makes it hard to justify significantly modifying any one of
the libraries. Ideally, the final product will have the following properties:
\begin{enumerate}
	\item minimal changes to the source code for each library;
	\item require building only once and only once; and
	\item reconfigurable at runtime.
\end{enumerate}

\newpage

To meet our design goals we propose a minimal programming interface in order to
make an application built using our framework TLS-aware. Each of the functions
defined here will support using any combination of the OpenSSL, LibreSSL, or
BoringSSL libraries as a backend.

\textbf{tls\_init}. Initializes the library, calling any initialize functions
that the backend requires.

\textbf{tls\_read}. Reads from a file descriptor.

\textbf{tls\_write}. writes to a file descriptor.

\textbf{tls\_deinit}. Cleans up and deinitializes the TLS libraries.

We explore how we have achieved each of our design goals in the next chapter.

\cleardoublepage

\chapter{Implementation}
\label{sec:impl}

This chapter is divided into three sections. In the first,
section~\ref{sec:scaff}, we discuss how the project is physically layed out --
the project scaffolding. The second section, section~\ref{sec:build}, documents
what kind of build system and modifications to the libraries' we had to employ
-- getting everything to play nicely.  The third section, section
~\ref{sec:symbol}, deals with issues that arose from linking multiple libraries
-- symbol resolution.

\section{Scaffolding}
\label{sec:scaff}

Installing OpenSSL~\cite{website:openssl}, BoringSSL~\cite{website:boringssl},
and LibreSSL~\cite{website:libressl} side-by-side on the same system can be a
daunting task for novices. Each of the libraries -- due to the nature of having
the same origin -- export similar symbols, build artifacts with the same name,
and install to the same directories.

Installation of any one implementation on a system-wide basis would mean that
another implementation cannot be used due to the resulting collisions. This
makes having an installation of more than one implementation based on OpenSSL
difficult, to say the least. These collisions make it impossible to link all
three into the same application without any edits to any of the implementations.

Our solution is to build each TLS implementation in a subdirectory, then link
them at runtime. This method gets rid of any high-level namespace collisions,
however, only works when we want to swap entire implementations out for each
other and therefore provides minimal modularity. The methods we used to achieve
this is discussed further in section~\ref{sec:build}, where we discuss our build
system. 

\section{Build infrastructure}
\label{sec:build}

Initially, we tried converting the build system for OpenSSL to CMake, the system
that BoringSSL and LibreSSL currently use. This proved to be challenging within
the two week time constraint and was dropped as a hard requirement for this
deliverable.

Additionally, OpenSSL requires many files to be generated before compilation and
a number of compiler flags to be passed in at configuration time. Many of these
flags are not documented, and as such are difficult to account for when
migrating the build system.

Additionally, to provide maximal modularity we must be able to swap exported
symbols at runtime. This issue is covered in-depth in section~\ref{sec:symbol}.

\newpage

\section{Symbol resolution}
\label{sec:symbol}

Explain what symbols are, and how resolution is done is.

Two of the libraries, BoringSSL and LibreSSL, export the same symbols for
various functions. Before we could have implemented a wrapper around each
library, we needed to ensure that the correct function would have been called
from each library. That is, we needed to make sure that the program, at runtime,
knew which version of a function to call.

Multiple solutions to get around symbol collisions were looked into, namely: the
use of objcopy~\cite{website:objcopy} to rename important symbols, writing a
LLVM compiler pass~\cite{website:llvmpass}, and using symbol
versioning~\cite{website:versioning}.

Ultimately, we moved forward with symbol versioning as it is, we believe, the
most transparent way to achieve non-colliding, globally exported symbols and
only requires minimal changes to each codebase. Our research into the
possibility of using each of of the proposed methods can be found in the next
section: section~\ref{sec:symbol_res}.

\subsection{Putting the ``solution'' to symbol resolution}
\label{sec:symbol_res}

\textbf{objcopy}. Using this method would require us to modify not only the
compiled shared object files, but also to create our own header files with
prototypes to each of the functions. This would, in essence, give a result
equivalent to writing our own compiler pass.

\textbf{LLVM compiler pass}. We would have to maintain our own header files.
This option is not entirely unreasonable, but it would require the maintenance
of many header files.

\textbf{Symbol versioning}. After reading the section on Symbol Relocations, and
Export Control of Drepper's
\textit{How To Write Shared Libraries}~\cite{website:howtodso}, we decided that
his method was perfect for our needs. It required, an additional two compiler
flags and the creation of a symbol map, which was trivial to write. By exporting
the version of a symbol, each library we linked to would have saved a reference
to the library it was referring to at link-time, which meant that at runtime the
expected functions should be called.

\cleardoublepage

\chapter{Evaluation}
\label{sec:eval}

We have tested all combinations of API libraries and none of them are
viable for practical use. Programs have either thrown a fatal error or have
leaked memory, not ideal in either case. The memory errors arise because of the
difference in the SSL\_CTX (ssl\_ctx\_st) structure defined in each library.

All combinations were, painstakingly, obtained by using both valgrind and gdb.

\begin{table}[h!]
	\begin{center}
		\begin{tabular}{ l | l | r }
			\textbf{init}       & \textbf{deinit} & \textbf{Result} \\
			\hline
			Open                & Open            & Works! \\
			Open                & Boring          & SIGABRT \\
			Open                & Libre           & SIGSEGV \\
			\rowcolor{gray!25}
			Boring              & Boring          & Works! \\
			\rowcolor{gray!25}
			Boring              & Open            & Leak! \\
			\rowcolor{gray!25}
			Boring              & Libre           & Leak! \\
			\rowcolor{gray!50}
			Libre               & Libre           & Works! \\
			\rowcolor{gray!50}
			Libre               & Boring          & SIGABRT \\
			\rowcolor{gray!50}
			Libre               & Open            & SIGSEGV \\
		\end{tabular}
		\caption{Tabulated results from testing.}
	\end{center}
\end{table}

It is unlikely that we will be able to use BoringSSL's SSL high-level
library in future versions, as it strays the most from OpenSSL. As noted
on the BoringSSL project page, ``Although BoringSSL is an open source project,
it is not intended for general use, as OpenSSL is. We don\'t recommend that
third parties depend upon it. Doing so is likely to be frustrating because there
are no guarantees of API or ABI stability.''\cite{website:boringssl}

There was no case in which any of the libraries worked well together, even with
some rudimentary "shim" code. The differences between each library's SSL
context, and what is required to correctly allocate it vary too greatly to be
able to be done in the time span of a sprint.

\cleardoublepage

\chapter{Conclusion}
\label{sec:con}

We have solved the namespace collision issues and have created a
proof-of-concept application which can swap out implementations of a function at
runtime. This deliverable provides a solid foundation for future work.

Further areas of work include increasing library modularity, and the using a
non-OpenSSL fork (e.g. GnuTLS).

Increased library modularity would allow for more than merely swapping
in--and--out implementations of TLS. Increased modularity in the HST library
could allow for the mixing and matching different TLS functions, such as init,
deinit, write, and read from different implementations.

HST could benefit greatly from using a completely different implementation of
TLS. A completely different implementation would reduce the amount of code
required to ensure no namespace collisions, however making sure that there is no
loss of data may be non-trivial. For example, the conversion of a OpenSSL TLS
structure to a GnuTLS TLS structure.

There is a lot of variance in how each library stores their respective SSL
context, notably BoringSSL. There is currently no safe combination of two
different libraries.

Future work should be done more than just the SSL-related parts of the
libraries, specifically the cryptography fundamentals of each library.

Areas for future research include the cryptography fundamentals that each
library provides, specifically their allocators, deallocators, locking, and
encryption APIs.

\cleardoublepage

\bibliographystyle{plain}
\bibliography{document.bib} 
\cleardoublepage

\appendix
\chapter{Referenced results}

\section{Example: linking with all 3 implementations}\label{app:linkwithimpl}

\begin{tiny}
\begin{verbatim}
$ LD_PRELOAD="lib/libhstopen_init.so lib/libhstboring_deinit.so lib/libhstlibre_version.so" ldd ./bin/init
        linux-vdso.so.1 (0x00007ffd6755a000)
        lib/libhstopen_init.so (0x00007f149ba99000)
        lib/libhstboring_deinit.so (0x00007f149b896000)
        lib/libhstlibre_version.so (0x00007f149b694000)
        libhststub.so => /.../hst/libhst/build/lib/libhststub.so (0x00007f149b492000)
        libc.so.6 => /lib64/libc.so.6 (0x00007f149b0a9000)
        libdl.so.2 => /lib64/libdl.so.2 (0x00007f149aea5000)
        libssl.so.1.1 => /.../hst/libhst/open/../../openssl/libssl.so.1.1 (0x00007f149ac3b000)
        libcrypto.so.1.1 => /.../hst/libhst/open/../../openssl/libcrypto.so.1.1 (0x00007f149a7df000)
        libssl.so => /.../hst/libhst/boring/../../boringssl/build/ssl/libssl.so (0x00007f149a596000)
        libcrypto.so => /.../hst/libhst/boring/../../boringssl/build/crypto/libcrypto.so (0x00007f149a1f7000)
        libcrypto.so.37 => /.../hst/libhst/libre/../../libressl/crypto/.libs/libcrypto.so.37 (0x00007f1499e1d000)
        /lib64/ld-linux-x86-64.so.2 (0x000055562a483000)
        libpthread.so.0 => /lib64/libpthread.so.0 (0x00007f1499bff000)
        libresolv.so.2 => /lib64/libresolv.so.2 (0x00007f14999e4000)
\end{verbatim}
\end{tiny}

\newpage

\section{Example: faking the library version}\label{app:fakeversion}

\begin{tiny}
\begin{verbatim}
$ LD_PRELOAD="lib/libhstopen_init.so lib/libhstopen_deinit.so lib/libhstlibre_version.so" ./bin/init
HST: using libre's impl of: hst_tls_version
Using version: LibreSSL 2.3.2
HST: using open's impl of: hst_tls_init
HST: using open's impl of: hst_tls_deinit
\end{verbatim}
\end{tiny}

\newpage

\end{document}
