\documentclass{acm_proc_article-sp}
\usepackage{graphicx}
\usepackage{caption}
\usepackage{subcaption}
\usepackage{appendix}
\usepackage{url}

\title{HotSwapTLS: Report for 29 January - 12 February, 2016}

\numberofauthors{1}
\author{
    \alignauthor
    Bheesham Persaud\\
        \affaddr{Carleton University}\\
        \affaddr{1125 Colonel By Drive}\\
        \affaddr{Ottawa, Ontario}\\
        \email{bheeshampersaud@cmail.carleton.ca}
}

\begin{document}

\maketitle

\begin{abstract}\label{sec:abstract}
  The init, deinit, and version functions were implemented during this sprint.
  There were a few challenges, mainly arising from the fact that each library
  exported the same symbols, and that each implementation differs so there are
  segfaults that need to be worked around.

  This deliverable is incomplete. It does not meet the requirements outlined in
  the proposal. The read and write functions are not implemented as there are
  still numerous design issues left to be ironed out. These issues will be
  expanded upon further. Additionally, testing the software is tedious as there
  are now $9$ combinations in which the implementations can be combined.

  Future work needs to be done on implementing more functions, both at a high
  level (e.g read, write), and at a low level (e.g. handling of data structures,
  specifically the SSL context). The handling of data structures must be done
  with care as it is trivially easy to introduce segfaults if done incorrectly.
\end{abstract}

\section{Introduction}

This report summarizes the work done between the period of 29 January 2016 and
12 February 2016. The goal for this two week sprint was to create an high level
API to each implementation, namely, OpenSSL~\cite{openssl},
LibreSSL~\cite{libre}, and BoringSSL~\cite{boringssl}.

The high level API was to expose functions to securely initiate, deinitialize,
read, and write session.

We encountered similar issues as we did in the previous deliverable, namely
collisions. This is discussed further in section~\ref{Sec:Issues}. Items
included in this deliverable are discussed in section~\ref{Sec:Deliv}.

\pagebreak

\section{Build issues}\label{Sec:Issues}

In the previous deliverable, we ran into namespace collisions yet again --
simply at a different level. Previously, namespace collisions happened on a
system-wide basis, that is, only one implementation was allowed to be installed.
In this deliverable, we encountered a different kind of namespace collision:
collisions between symbols that are exported by the libraries. Symbol collision
is discussed further in section~\ref{Sec:Symbol}.

Additionally, the swapping of implementations of the high-level interfaces is
unsupported at this time due to the differences in how each library does each
operation. Operations on a data structure must be done using functions from the
same library otherwise segfaults will be introduced into the program -- the
natural enemy of secure software. Segmentation faults were completely expected,
and will be ironed out as the project matures.

\subsection{Symbol Collisions}\label{Sec:Symbol}

Two the libraries, specifically BoringSSL and LibreSSL, export the same symbols
for various functions. Before we could have implemented a wrapper around each
library, we needed to ensure that the correct function would have been called
from each library. That is, we needed to make sure that the program, at runtime,
knew which version of a function to call.

Multiple solutions to get around symbol collisions were looked into, namely: the
use of objcopy~\cite{objcopy} to rename important symbols, writing a LLVM
compiler pass~\cite{llvmpass}, and using symbol versioning~\cite{versioning}.

Ultimately, we moved forward with symbol versioning as it is, we believe, the
most transparent way to achieve non-colliding, globally exported symbols and
only requires minimal changes to each codebase. Our research into the
possibility of using each of of the proposed methods can be found in the next
section: section~\ref{Sec:Res}.

\subsubsection{Results}\label{Sec:Res}

\textbf{objcopy}. Using this method would require us to modify not only the
compiled shared object files, but also to create our own header files with
prototypes to each of the functions. This would, in essence, give a result
equivalent to writing our own compiler pass.

\textbf{LLVM compiler pass}. We would have to maintain our own header files.
This option is not entirely unreasonable, but it would require the maintenance
of many header files.

\textbf{Symbol versioning}. After reading the section on Symbol Relocations, and
Export Control of Drepper's
\textit{How To Write Shared Libraries}~\cite{howtodso}, we decided that his
method was perfect for our needs. It required, an additional two compiler flags
and the creation of a symbol map, which was trivial to write. By exporting the
version of a symbol, each library we linked to would have saved a reference to
the library it was referring to at link-time, which meant that at runtime the
expected functions should be called.

\section{Deliverable}\label{Sec:Deliv}

This deliverable adds the ability for a user to report using a fake library, as
mixing and matching the high level API (init, deinit, read, write) are not yet
supported.

Examples can be seen in appendicies~\ref{Ex:ex1} and~\ref{Ex:ex2}.

\section{Conclusion}

Symbol collisions are a problem of the past thanks to symbol versioning. There
was no need to use binutils, nor to create a LLVM compiler pass in order to make
sure there are no symbol collisions.

There is, however, still the issue of segmentation faults due to how each
library implements each function.

Future work includes adding more high level and low level function calls to
libhst from all 3 implementations, and to make sure that the library exposed by
libhst is safe and bug-free. In order to deliver a stable version of libhst, I
propose that development on the binary utility to aid in preloading libraries at
runtime be postponed.

\bibliographystyle{abbrv}
\bibliography{sigproc}

\onecolumn
\newpage
\appendix
\section{Example: linking with all 3 implementations}\label{Ex:ex1}
\begin{verbatim}
$ LD_PRELOAD="lib/libhstopen_init.so lib/libhstboring_deinit.so lib/libhstlibre_version.so" ldd ./bin/init
        linux-vdso.so.1 (0x00007ffd6755a000)
        lib/libhstopen_init.so (0x00007f149ba99000)
        lib/libhstboring_deinit.so (0x00007f149b896000)
        lib/libhstlibre_version.so (0x00007f149b694000)
        libhststub.so => /.../hst/libhst/build/lib/libhststub.so (0x00007f149b492000)
        libc.so.6 => /lib64/libc.so.6 (0x00007f149b0a9000)
        libdl.so.2 => /lib64/libdl.so.2 (0x00007f149aea5000)
        libssl.so.1.1 => /.../hst/libhst/open/../../openssl/libssl.so.1.1 (0x00007f149ac3b000)
        libcrypto.so.1.1 => /.../hst/libhst/open/../../openssl/libcrypto.so.1.1 (0x00007f149a7df000)
        libssl.so => /.../hst/libhst/boring/../../boringssl/build/ssl/libssl.so (0x00007f149a596000)
        libcrypto.so => /.../hst/libhst/boring/../../boringssl/build/crypto/libcrypto.so (0x00007f149a1f7000)
        libcrypto.so.37 => /.../hst/libhst/libre/../../libressl/crypto/.libs/libcrypto.so.37 (0x00007f1499e1d000)
        /lib64/ld-linux-x86-64.so.2 (0x000055562a483000)
        libpthread.so.0 => /lib64/libpthread.so.0 (0x00007f1499bff000)
        libresolv.so.2 => /lib64/libresolv.so.2 (0x00007f14999e4000)
\end{verbatim}

\section{Example: faking the library version}\label{Ex:ex2}
\begin{verbatim}
$ LD_PRELOAD="lib/libhstopen_init.so lib/libhstopen_deinit.so lib/libhstlibre_version.so" ./bin/init
HST: using libre's impl of: hst_tls_version
Using version: LibreSSL 2.3.2
HST: using open's impl of: hst_tls_init
HST: using open's impl of: hst_tls_deinit
\end{verbatim}

\balancecolumns

\end{document}
