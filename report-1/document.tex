\documentclass{acm_proc_article-sp}
\usepackage{graphicx}
\usepackage{caption}
\usepackage{subcaption}
\usepackage{url}

\title{HotSwapTLS: Report for January 18-28, 2016}

\numberofauthors{1}
\author{
    \alignauthor
    Bheesham Persaud\\
        \affaddr{Carleton University}\\
        \affaddr{1125 Colonel By Drive}\\
        \affaddr{Ottawa, Ontario}\\
        \email{bheeshampersaud@cmail.carleton.ca}
}

\begin{document}

\maketitle

\begin{abstract}\label{sec:abstract}
  Installing BoringSSL~\cite{boringssl}, LibreSSL~\cite{libressl}, and
  OpenSSL~\cite{openssl} side-by-side on the same system requires an unfeasible
  amount of work.

  A major source of issues arose from the the fact that modifying each of the
  projects' build system to ensure the required include files, the linked
  libraries, and the compiled symbols will not collide would have taken a
  non-trivial amount of time. Additionally, major issues also arose due to
  OpenSSL's inclusion of legacy and, in some cases, vestigial code.

  Our solution is to build each TLS implementation in a subdirectory, then link
  to them. This method gets rid of any high-level namespace collisions, however,
  only works when we want to swap entire implementations out for each other and
  therefore provides minimal modularity.

  Needless to say, building each dependency in accordance to the original
  project proposal did not go according to plan, but contributions were
  suggested and accepted upstream -- everybody wins.
\end{abstract}

\section{Introduction}

This report summarizes the work done between the period of 18 January 2016 and
29 January 2016. The goal for this two week sprint was to correctly build each
dependency (BoringSSL, LibreSSL, and OpenSSL) and set up the scaffolding for
future work.

\section{Build issues}

There were two major issues encountered with this deliverable. The first major
issue stems from the fact that both BoringSSL and LibreSSL are forks of OpenSSL,
that is to say we encountered numerous namespace collisions.

The second major issue arose from how much legacy code OpenSSL has. The legacy
code made migrating build systems unfeasible for this deliverable.

\subsection{Namespace collisions}

Each of the libraries export similar symbols,
build artifacts with the same name, and install to the same directories, and as
such it would be difficult to link all three into the same application without
any edits to any of the implementations.

The implementations all expect that their respective header files will be
located at $/usr/include/openssl$, and their libraries to be named $libcrypt$
and $libssl$. This makes having an installation of more than one implementation
based on OpenSSL difficult to say the least.

\subsection{Legacy code in OpenSSL}

Initially, we tried converting the build system
for OpenSSL to CMake, the system that BoringSSL and LibreSSL currently use. This
proved to be challenging within the two week time constraint and was dropped as
a hard requirement for this deliverable.

OpenSSL currently includes an unreasonable amount of legacy code in its build
scripts. For example, there were multiple tests that were not being used, or
built that were included in the source~\cite{issue}.

Additionally, OpenSSL requires many files to be generated before compilation and
a number of compiler flags to be passed in at configuration time. Many of these
flags are not documented, and as such are difficult to account for when
migrating the build system.

\section{Solution}

Each of the implementations, BoringSSL; LibreSSL; and OpenSSL, build target
libraries $libcrypt$ and $libssl$, and place their header files in
$/usr/include/openssl$. Installation of any one TLS implementation on a
system-wide basis would mean that another implementation cannot be used due to
the resulting collisions.

HST requires that each of the TLS implementations be available on the system,
and as such scaffolding has been set up so that each of the dependencies are
built in-source. In the future, this may be changed to a subdirectory in $/var$
or perhaps a folder in the user's home directory.

\section{Conclusion}

Further areas of work include increased library modularity, usage of a
non-OpenSSL fork (e.g. GnuTLS).

Increased library modularity would allow for more than merely swapping
in--and--out implementations of TLS. Increased modularity in the HST library
could allow for the mixing and matching different TLS functions, such as init,
deinit, write, and read from different implementations.

HST could benefit greatly from using a completely different implementation of
TLS. A completely different implementation would reduce the amount of code
required to ensure no namespace collisions, however making sure that there is no
loss of data may be non-trivial. For example, the conversion of a OpenSSL TLS
structure to a GnuTLS TLS structure.

\bibliographystyle{abbrv}
\bibliography{sigproc}

\balancecolumns

\end{document}
