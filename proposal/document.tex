\documentclass{acm_proc_article-sp}
\usepackage{graphicx}
\usepackage{caption}
\usepackage{subcaption}
\usepackage{url}

\title{HotSwapTLS}

\numberofauthors{1}
\author{
    \alignauthor
    Bheesham Persaud\\
        \affaddr{Carleton University}\\
        \affaddr{1125 Colonel By Drive}\\
        \affaddr{Ottawa, Ontario}\\
        \email{bheeshampersaud@cmail.carleton.ca}
}

\begin{document}

\maketitle

\begin{abstract}\label{sec:abstract}

The purpose of the HotSwapTLS project is to create a method in which
applications can use different implementation of Transport Layer Security (TLS)
without requiring recompilation. The TLS implementations that we will support
include: OpenSSL~\cite{openssl}, LibreSSL~\cite{libressl}, and
BoringSSL~\cite{boringssl}.

This will be achieved by building applications which link against a stub
library, $libhst$. A minimal TLS API will be exposed by different libraries
which can be loaded at runtime. Each library will will use a different TLS
implementation as it's backend.

Additionally, $hst$, a tool to easily manage different TLS implementations and
their versions will be created. This is required as the TLS implementations are
often not compatible with each other -- installing one overwrites the other, as
is the case with LibreSSL and OpenSSL.

\end{abstract}

\section{Motivation}\label{Sec:Motiv}

The majority of networking software built require, more or less, the same
protocols to be implemented. Chances are, if you're running a server application
then it requires library which implements the Transport Layer Security (TLS
protocol) -- a protocol that is notoriously hard to implement correctly. There
are multiple security vulnerabilities reported in implementations of TLS each
year.

In 2014 and 2015 the OpenSSL project reported a total of 54 Common
Vulnerabilities and Exposures (CVE)~\cite{opensslsec}, of which the infamous
Heartbleed bug was included.

The announcement of the Heartbleed bug in OpenSSL caused many community driven
forks to be created, including LibreSSL and BoringSSL. LibreSSL is a fork
managed by the OpenBSD developers and BoringSSL is a fork managed by Google.

\pagebreak

\section{Objectives}\label{Sec:Obj}

The final deliverable of this project is a run-time environment which allows
users to be able to run applications with different security libraries without
having to go through the trouble of recompilation.

HotSwapTLS is an ambitious project and requires multiple deliverables to be
completed.

\subsection{Expected deliverables}\label{Sec:Deliver}

There will be two deliverables from this project. $libhst$, the core library
which applications need to be built with; and $hst$, a utility for managing TLS
backends.

A timeline in which the $libhst$ and $hst$ deliverables can be expected is outlined
in section~\ref{Sec:Time}.

\subsubsection{libhst}

$libhst$ is the core of the project and will require the bulk of work. $libhst$
will define a minimal programming interface in order to make an application
TLS-aware. Each of the functions defined here will have to support using either
of the TLS libraries as a backend, which, for this project, will be OpenSSL,
LibreSSL, and BoringSSL.

\textbf{tls\_init}. Initializes the library, calling any initialize functions
that the backend requires.

\textbf{tls\_read}. Reads from a file descriptor.

\textbf{tls\_write}. writes to a file descriptor.

\textbf{tls\_deinit}. Cleans up and deinitializes the TLS libraries.

\subsubsection{hst}

$hst$ will be a binary utility to manage installed TLS backends. Currently,
popular operating systems, such as Fedora, do not package OpenSSL and it's forks
in a way which they are compatible with each other. For example, an installation
of LibreSSL will overwrite a previous installation of OpenSSL and vice versa.

$hst$ will also be able to manage different versions of the same library. This
is to mitigate bugs introduced by updates.

\pagebreak

\section{Timeline}\label{Sec:Time}

Development of HotSwapTLS will take place over the course of 12 weeks. The
project's expected timeline as well as expected completion dates of deliverables
such as $libhst$, $hst$, and a final report are outlined below.

\subsection{January, 2016}

\textbf{Week 02 (Jan. 25 -- Jan. 29).} Project scaffolding to build OpenSSL,
LibreSSL, BoringSSL, stub libraries, and API implementations.

\subsection{February, 2016}

\textbf{Week 04 (Feb. 08 -- Feb. 12).} $libhst$: a collection of the stub
library applications will link against and various libraries that implement a
minimal TLS API using the different TLS backends.

\textbf{Week 06 (Feb. 22 -- Feb. 26).} Prepare mid-term report and implement an
echo server and client to test $libhst$.

\subsection{March, 2016}

\textbf{Week 08 (Mar. 07 -- Mar. 11).} Begin work on the final report and test
the libraries.

\textbf{Week 10 (Mar. 21 -- Mar. 25).} Implement $hst$: an application to make
running $libhst$-compatible applications easier.

\subsection{April, 2016}

\textbf{Week 12 (Apr. 04 -- Apr. 08).} Preparation of final report.

\bibliographystyle{abbrv}
\bibliography{sigproc}

\balancecolumns

\end{document}
