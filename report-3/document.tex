\documentclass{acm_proc_article-sp}
\usepackage{appendix}
\usepackage{caption}
\usepackage{graphicx}
\usepackage{subcaption}
\usepackage{url}
\usepackage[table]{xcolor}

\title{HotSwapTLS: Report for 13 February - 26 February, 2016}

\numberofauthors{1}
\author{
    \alignauthor
    Bheesham Persaud\\
        \affaddr{Carleton University}\\
        \affaddr{1125 Colonel By Drive}\\
        \affaddr{Ottawa, Ontario}\\
        \email{bheeshampersaud@cmail.carleton.ca}
}

\begin{document}

\maketitle

\begin{abstract}\label{sec:abstract}

  The three different libraries have varying representations of how to store
  global SSL session data -- that is, data that is required to initiate a secure
  connection, including the CA certificate, private key, and various
  configuration options.

  The method we chose to combine the different libraries, as it turns out, was
  not a great choice. The SSL contexts between the different libraries differ
  too much to be able to replace the higher-level APIs, but, replacing the
  fundamental cryptography APIs could yield better results.

  Future work should be done more than just the SSL-related parts of the
  libraries, specifically the cryptography fundamentals of each library.

\end{abstract}

\section{Introduction}

This report summarizes the work done between the period of 13 February 2016 and
26 February 2016. The goal of this sprint was to debug the memory errors that
were being reported when the different libraries of TLS were mixed.  No further
work was done on adding high level APIs to read or write data from sessions
apart getting rid of bugs.

The differences between OpenSSL, BoringSSL, and LibreSSL were painfully apparent
for the duration of this sprint. As is well documented by Google, BoringSSL does
not try to be a drop-in replacement for OpenSSL.

In this report, we investigate the differences between each of the contexts and
propose possible techniques for mitigation. Additionally, we raise some security
concerns with regards to dynamically loading different libraries of TLS.

\section{Testing configurations}\label{Sec:Config}

We have tested all combinations of API libraries and none of them are
viable for practical use. Programs have either thrown a fatal error or have
leaked memory, not ideal in either case. The memory errors arise because of the
difference in the SSL\_CTX (ssl\_ctx\_st) structure defined in each library.

All combinations were, painstakingly, obtained by using both valgrind and gdb.

\subsubsection{Results}\label{Sec:Res}

\begin{table}
\begin{center}
    \begin{tabular}{ l | l | r }
      \textbf{init}       &       \textbf{deinit} &       \textbf{Result}  \\
      \hline
      Open                &       Open            &       Works!\\
      Open                &       Boring            &       SIGABRT\\
      Open                &       Libre            &       SIGSEGV\\
      \rowcolor{gray!25}
      Boring                &       Boring            &       Works!\\
      \rowcolor{gray!25}
      Boring                &       Open            &       Leak!\\
      \rowcolor{gray!25}
      Boring                &       Libre            &       Leak!\\
      \rowcolor{gray!50}
      Libre                &       Libre            &       Works!\\
      \rowcolor{gray!50}
      Libre                &       Boring            &       SIGABRT\\
      \rowcolor{gray!50}
      Libre                &       Open            &       SIGSEGV\\
    \end{tabular}
    \caption{Tabularized results from testing.}
\end{center}
\end{table}

It is unlikely that we will be able to use BoringSSL's SSL high-level
library in future versions, as it strays the most from OpenSSL. As noted
on the BoringSSL project page, ``Although BoringSSL is an open source project,
it is not intended for general use, as OpenSSL is. We don\'t recommend that
third parties depend upon it. Doing so is likely to be frustrating because there
are no guarantees of API or ABI stability.''\cite{boringssl}

There was no case in which any of the libraries worked well together, even with
some rudimentary "shim" code. The differences between each library's SSL
context, and what is required to correctly allocate it vary too greatly to be
able to be done in the time span of a sprint.

\section{Deliverable}\label{Sec:Deliv}

This deliverable includes only bug fixes and analysis.

\section{Conclusion}

There is a lot of variance in how each library stores their respective SSL
context, notably BoringSSL. There is currently no safe combination of two
different libraries.

Future work should be done more than just the SSL-related parts of the
libraries, specifically the cryptography fundamentals of each library.

Areas for future research include the cryptography fundamentals that each
library provides, specifically their allocators, deallocators, locking, and
encryption APIs.

\bibliographystyle{abbrv}
\bibliography{sigproc}
\balancecolumns
\end{document}
